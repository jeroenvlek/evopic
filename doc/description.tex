\documentclass[a4paper, 11pt, onecolumn]{article}
\author{Jeroen Vlek\\www.perceptivebits.com\\jeroenvlek@gmail.com}
\title{EvoPic}


\usepackage{amsmath}
\usepackage{amssymb}
\usepackage{fullpage}
\usepackage{graphicx}
\usepackage{subfigure}
\usepackage{setspace}
\usepackage{float}
\usepackage{color}
\usepackage{listings}
\usepackage[boxed]{algorithm2e}

\newcommand{\ds}{\displaystyle}
\newcommand{\todo}[1]{\textcolor{red}{\textbf{TODO: #1}}}

\begin{document}

\abstract{This document provides a sketch for the EvoPic program. The purpose of this 
document is to serve as a basis for setting up the class structure of EvoPic.}

\section{Introduction}
 
EvoPic is a program that approximates a target image using polygons of different
shapes, sizes and colors. It does this by using a genetic algorithm that evolves
a population of different organisms. An organism is a set of polygons drawn on a
canvas the same size as the target image, where the polygons are described by
the organism's "DNA". Two organisms can have offspring, where a child receives
its DNA from its two parents (note: the organisms are genderless). DNA is
further altered by random mutation. Organisms will also die due through natural
selection. Each organism is compared to the original image, the organisms that
fit the image the least are killed.

The algorithm starts with a random initialization of an even number of organisms
and then enters the following loop:

\begin{itemize}
\item Randomly pair up all organisms
\item Create one child for each pair, resulting in $n$ children
\item Randomly mutilate the DNA of all children
\item Natural selection: Remove the $n$ organisms that are the least like the
target image 
\item Repeat until user stops loop
\end{itemize}

The design of EvoPic will follow a variant of the \textit{Model-View-Controller}
(MVC) architectural design paradigm, where the \textit{Model} is the population,
the \textit{View} visualizes the current state of the population and the
\textit{Controller} is the genetic algorithm. The difference with the
actual MVC pattern is that the genetic algorithm is an ongoing process and does
not require user input (apart from stopping and starting the loop), i.e. it is
not a state machine.

Given the above, a preposition of classes can be made:

Model: 
Population
Organism 
DNA
Polygon
TargetImage

Controller:
GeneticAlgorithm
Pairing
ImageCompare
QThread

View:
Image
TargetImage
PhenotypeImage
QMainWidget
QInputDialog
QButton



\end{document}